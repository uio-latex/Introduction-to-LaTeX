% Compile the document using:
%
%    pdflatex <filename>.tex
%
% If you need to use a system font, compile with one of the following:
%
%    xelatex <filename>.tex
%    lualatex <filename>.tex
%
% XeLaTeX and LuaLaTeX are newer, but pdfLaTeX is still the standard.


\documentclass[a4paper]{memoir}


% Common document classes:
% article - For short papers, 1-20 pages
% amsart  - Similar to article, following the style guide by AMS
% report  - For longer reports, 20-80 pages
% book    - For even longer documents, 80+ pages
% memoir  - A very customisable class that can play all of the above roles
% beamer  - For slides


% Some document class options:
% a4paper             - Paper format
% 11pt / 12pt         - Font size
% draft               - Highlight overfull hboxes
% oneside / twoside   - Margins
% openany / openright - Chapter starting page
% fleqn               - Left-justify display style math
% leqno               - Put equation numbers on the left
% article             - Imitate the Article class with Memoir


\usepackage[utf8]{inputenx}     % Source file encoding
\usepackage[T1]{fontenc}        % Font encoding
\usepackage{microtype}          % Improved typography
\usepackage{amssymb}            % Extra symbols
\usepackage{mathtools}          % Math fonts and more
\usepackage{amsthm}             % Theorem-like environments
\usepackage{thmtools}           % Theorem-like environments, extends amsthm
\usepackage{graphicx}           % Tool for images
\usepackage{varioref}           % Cross references, \vref
\usepackage{hyperref}           % Clickable links
\usepackage[nameinlink,
            capitalize,
            noabbrev]{cleveref} % Cross references, \cref


\DeclareMathOperator{\rank}{rank}
\DeclarePairedDelimiter{\p}{(}{)}
\declaretheorem[style = plain, numberwithin = chapter]{theorem}


\begin{document}


\chapter{Basic Syntax}


\LaTeX\ commands begin with a backslash (there are some exceptions in old \TeX). They come in two flavours, simple macros and environments. Their syntax is as follows:

% \macro[optional argument]{input}

% \begin{environment}[optional argument]
%     input
% \end{environment}

If a macro is not followed by curly braces, then it will use the next non-space character as its input. \textbf Example vs \textbf{Example}. Macros that do not take any input will also swallow all space following it. Consider for instance the \LaTeX logo. To stop this, the macro can be closed by either a backslash or empty braces (\LaTeX\ or \LaTeX{}).

The number of spaces in the source code          does        not        matter.
Neither does a single line break.

But two line breaks start a new paragraph.


\chapter{Mathematics}


\section{Inline and Display Style}
Inline math mode is enabled with \( \) (\LaTeX) or $ $ (\TeX). This is used for single variables, such as \(x\), or small expressions, such as
\(
    \cos \frac{\pi}{2} + i\sin \frac{\pi}{2} = i
\),
inside normal text.

Display style mathematics is (typically) given a line of its own, and is used for long expressions or to highlight a certain formula. There are numerous ways to enable display style. We will only mention a few. The analogous environments to the inline math modes are
\[
    \lim_{n \to \infty} a_n
\]
and
$$
    f \colon X \to Y
$$
The drawbacks of these are that they only allow a single line and they have no counter. The \emph{equation} environment also allows only one line, but it has a counter that is removed in the starred version.
\begin{equation}
    \label{eq:matrix}
    \begin{bmatrix*}[r]
        1 & 0 & -1 \\
        0 & 1 &  0 \\
        1 & 0 &  1
    \end{bmatrix*}
\end{equation}
\begin{equation*}
    \binom{n}{k} = \frac{n!}{k! (n - k)!}
\end{equation*}
The \emph{align} environment allows for multiple lines. Again, the star removes the counters.
\begin{align*}
    \int \frac{2x + 3}{(x - 2)(x + 2)} \mathrm{d}x
    &=
    \int \frac{7}{4(x - 2)} + \frac{1}{4(x + 2)} \mathrm{d}x
    \\
    &=
    \frac{7}{4} \ln |x - 2| + \frac{1}{4} \ln |x + 2| + \mathcal{C}
\end{align*}
The double backslashes used to break the line is a normal macro that can take an optional argument. This argument is a length, that can be negative, which is added to the space between the lines:
\begin{align*}
    \int \frac{2x + 3}{(x - 2)(x + 2)} \mathrm{d}x
    &=
    \int \frac{7}{4(x - 2)} + \frac{1}{4(x + 2)} \mathrm{d}x
    \\[5ex]
    &=
    \frac{7}{4} \ln |x - 2| + \frac{1}{4} \ln |x + 2| + \mathcal{C}
\end{align*}
% Some units:
% ex - Height of small 'x'
% em - Width of capital 'M'
% pt - Point, 1/72.27 inch
% cm
% mm
% \textwidth
% \textheight
It is possible to force display style in inline math, but it is generally not recommended since it adds too much white space between the lines.
\(
    \sum_{i = 1}^{n} x_i
\)
vs 
\(
    \displaystyle
    \sum_{i = 1}^{n} x_i
\).
Some commands have special display style versions. \(\dfrac{1}{2}\) vs \(\frac{1}{2}\).
\begin{align*}
    f
    &=
    \begin{dcases*}
        \frac{1}{2} & if \(x \in [0, 1]\)
        \\
        1 & if \(x > 1\)
    \end{dcases*}
    \shortintertext{vs.}
    f
    &=
    \begin{cases*}
        \frac{1}{2} & if \(x \in [0, 1]\)
        \\
        1 & if \(x > 1\)
    \end{cases*}
\end{align*}


\section{Custom Macros}
In math mode, all letters are assumed to be variables. This means that they are italicised, but it also means extra space is added around each letter. Notice the difference:
\begin{gather*}
    MATHEMATICS
    \shortintertext{vs.}
    \mathit{MATHEMATICS}
\end{gather*}
The most common way to write normal text in math mode is with \emph{operators}. These are written in an upright font, followed by a small space. Notice the difference between just an upright font, \(\mathrm{sin} x\), and a predefined operator, \(\sin x\). We can declare our own operators. In the preamble we have defined the rank operator \(\rank A\) by
% \DeclareMathOperator{\rank}{rank}

The parentheses in
\begin{align*}
    (\sum_{i = 1}^{n} x_i)
\end{align*}
look too small. We can fix this in this way:
\begin{align*}
    \Bigg(\sum_{i = 1}^{n} x_i\Bigg)
\end{align*}
We had to manually check which macro gave the correct size. An automatic solution to this is the following:
\begin{align*}
    \left(\sum_{i = 1}^{n} x_i\right)
\end{align*}
After a while, we get tired of writing left and right. We can declare parentheses as paired delimiters in the preamble with
% \DeclarePairedDelimiter{\p}{(}{)}
We can then use the newly defined command as follows:
\begin{align*}
    \p{\sum_{i = 1}^{n} x_i}
    \\
    \p[\Bigg]{\sum_{i = 1}^{n} x_i}
    \\
    \p*{\sum_{i = 1}^{n} x_i}
\end{align*}
Note that you cannot use left and right across multiple lines! In that case you need to control the size manually.

We also have the \emph{middle} macro, but not all symbols can use it. The normal set builder symbol is \(\mid\). Note that \(|\) has too little white space around it. However, \emph{mid} does not scale, so the best we can achieve is
\begin{align*}
    \left\{
        \mathbf{x} \in \mathbb{R}^n
        \mid
        \sum_{i = 1}^{n} = 1
    \right\}
\end{align*}
However, we can do the following:
\begin{align*}
    \left\{
        \mathbf{x} \in \mathbb{R}^n
        \middle\arrowvert
        \sum_{i = 1}^{n} = 1
    \right\}
\end{align*}

We can define environments for theorem-like statements using the \emph{amsthm} and \emph{thmtools} packages in this manner:
% \declaretheorem[style = plain, numberwithin = chapter]{theorem}

\begin{theorem}[Famous Mathematician]
    True statement.
\end{theorem}

\begin{proof}
    Reasons.
\end{proof}

Use custom macros to save typing! It is boring to write \(\mathbb{C}\) every time. With
\newcommand{\C}{\mathbb{C}}
we only need to type \(\C\).


\section{Miscellaneous}
Bold symbols come in two flavours. The most common is bold and upright, such as \(\mathbf{a}\). This is, among other things, used to denote vectors. The other flavour looks like the normal math symbols, just drawn thicker, such as \(\boldsymbol{a}\). This is nice for table and section headings.

Below are various much-used symbols:

Ellipses:
\(a_1, \ldots, a_n\) and \(a_1 \times \cdots \times a_n\)

Implications: \(\implies, \impliedby, \iff\)

Set theory:
\(
    \cup, \cap, \setminus, \subset, \subseteq, \supset, \supseteq,
    \in, \ni, \notin, \not\ni, \emptyset, \varnothing
\)

Relation symbols: \(\le, \leqslant, \ge, \geqslant\)

Various: \(\forall, \exists, \nexists, \pm, \mp\)

Greek letters:
\begin{align*}
    \alpha, \beta, \gamma, \delta, \epsilon, \varepsilon, \zeta,
    \eta, \theta, \vartheta, \iota, \kappa, \lambda, \mu, \nu,
    \xi, o, \pi, \varpi, \rho, \varrho, \sigma, \varsigma, \tau,
    \upsilon, \phi, \varphi, \chi, \psi, \omega
    \\
    A, B, \Gamma, \Delta, E, Z, H, \Theta, I, K, \Lambda, M, N, \Xi,
    O, \Pi, P, \Sigma, T, \Upsilon, \varUpsilon, \Phi, X, \Psi, \Omega
\end{align*}


\chapter{Floats and Cross References}


\section{Figures}
Images can be included like this:

\includegraphics{apollon}

\noindent
We can control the size in various ways:

\includegraphics[height = 0.2\textheight]{apollon}
\includegraphics[width = 0.3\textwidth]{apollon}
\includegraphics[scale = 0.5]{apollon}

\noindent
We can centre the image and add some white space before and after the image:
\begin{center}
    \includegraphics[scale = 0.5]{apollon}
\end{center}
Images included like this will stay exactly where we placed them. However, the \LaTeX\ way is to wrap images in a \emph{floating} environment. This allows \LaTeX\ to place them wherever \LaTeX\ thinks it is best in respect to page and paragraph breaks. For images, the correct float is \emph{figure}:

\begin{figure}[h!tbp]
    \centering
    \includegraphics{apollon}
    \caption{Apollon seal belonging to the UiO logo}
    \label{fig:apollon}
\end{figure}

You can optionally specify a prioritised list of possible positions for the float. The options are `h' for \emph{here}, `t' for the \emph{top} of a page, `b' for the \emph{bottom} of a page and `p' for a \emph{page} consisting only of floats. Adding an exclamation mark `!' to the list allows \LaTeX\ to slightly alter the amount of white space around the float in order to achieve the desired position.


\section{Tables}


Similarly, tables can also be placed anywhere with \emph{tabular}:

\begin{center}
    \begin{tabular}{llr}
        \toprule
        \multicolumn{2}{c}{Item} &            \\
        \cmidrule(r){1-2}
        Animal    & Description  & Price (\$) \\
        \midrule
        Gnat      & per gram     & 13.65      \\
                  & each         & 0.01       \\
        Gnu       & stuffed      & 92.50      \\
        Emu       & stuffed      & 33.33      \\
        Armadillo & frozen       & 8.99       \\
        \bottomrule
    \end{tabular}
\end{center}
However, it is best to put the tabular inside the float \emph{table}.

\begin{table}[p]
    \centering
    \begin{tabular}{llr}
        \toprule
        \multicolumn{2}{c}{Item} &            \\
        \cmidrule(r){1-2}
        Animal    & Description  & Price (\$) \\
        \midrule
        Gnat      & per gram     & 13.65      \\
                  & each         & 0.01       \\
        Gnu       & stuffed      & 92.50      \\
        Emu       & stuffed      & 33.33      \\
        Armadillo & frozen       & 8.99       \\
        \bottomrule
    \end{tabular}
    \caption{Price of animals}
    \label{tab:animals}
\end{table}


\section{Cross References}
With floats, you cannot write statements like `as can be seen in the following figure/table'. Instead, you should write `as can be seen in Figure~1/Table~2.3'.

Anything with a counter (chapters, sections, equations, theorems, figures, tables, etc.) can be given a label key with \label{key}. The counter can then be referenced with \ref{key}. For equations, it is better to use \eqref{eq:matrix}.

Using the \emph{cleveref} package, we get the macro \cref{key}, which will print what type of object the key refers to. In addition, the macro \vref{tab:animals} from the \emph{varioref} package will print where the object is.



\chapter{General Writing}


\section{Fonts}
To \emph{emphasise} text, use the \emph{emph} macro. \emph{This is a stackable macro that produces italics in an \emph{upright} environment and \emph{upright} font in an italicised environment.} Here are other macros for manipulating text\footnote{This is not a properly typeset document. Just because I have created a non-floating table, does not mean that you should.}:

\begin{center}
    \begin{tabular}{ll}
        \textit{Italics}    & \(\mathit{Italics}\)    \\
        \textsl{Slanted}    &                         \\
        \textbf{Bold face}  & \(\mathbf{Bold face}\)  \\
        \textsc{Small caps} &                         \\
        \texttt{Teletype}   & \(\mathtt{Teletype}\)   \\
        \textsf{Sans serif} & \(\mathsf{Sans serif}\)
    \end{tabular}
\end{center}


\section{Lists}
Lists can be created in the following ways:

\begin{itemize}
    \item
    This is a list

    \item
    with bullet points.
\end{itemize}

\begin{enumerate}
    \item
    This is a

    \item
    numbered list.
\end{enumerate}

\begin{description}
    \item[Description]
    is a list that highlights the word that is being defined or described. The rest of the text is indented.
\end{description}

Special list can be automatically created (must be compiled twice):

\tableofcontents
\listoffigures
\listoftables


\chapter{Useful Packages}


(That are not included in the preamble.)

\begin{description}
    \item[babel]
    Automatic translations

    \item[biblatex]
    Bibliography tool

    \item[listings]
    Typesetting of code

    \item[tikz]
    General drawing tool (If you do not want to learn Ti\textit{k}Z, then you can export code from GeoGebra)

    \item[tikz-cd]
    Drawing tool for commutative diagrams

    \item[textcomp]
    Extra symbols

    \item[siunitx]
    Typesetting of units

    \item[todonotes]
    Marginal notes

    \item[showkeys]
    Labels in margins

    \item[enumitem]
    Provides options for the list environments. This will overwrite the \emph{enumerate} package that is already loaded by Memoir.
\end{description}


\chapter{Learn More}


\begin{description}
    \item[Package documentations]
    \url{http://ctan.org}

    \item[UiO \LaTeX\ resources]
    \url{http://www.mn.uio.no/ifi/tjenester/it/hjelp/latex/}

    \item[General introduction to \LaTeX]
    \url{https://en.wikibooks.org/wiki/LaTeX}

    \item[Find symbols]
    \url{http://detexify.kirelabs.org/classify.html}

    \item[Ti\textit{k}Z examples with source code]
    \url{http://www.texample.net/tikz/}

    \item[Ti\textit{k}Z-CD editor]
    \url{https://tikzcd.yichuanshen.de/}

    \item[Available fonts]
    \url{http://www.tug.dk/FontCatalogue/}

    \item[Ask for help]
    \url{http://tex.stackexchange.com/}
\end{description}


\end{document}
